% This Document is Copyright 2018 the authors.

\documentclass[12pt, modern]{aastex62}

\addtolength{\topmargin}{-0.25in}
\addtolength{\textheight}{0.50in}
\setlength{\parindent}{\baselineskip}

\newcommand{\acronym}[1]{{\small{#1}}}
\newcommand{\Gaia}{\textsl{Gaia}}
\newcommand{\lcdm}{\acronym{$\Lambda$CDM}}
% \newcommand{\Tycho}{\textsl{Tycho}}
% \newcommand{\DRone}{\textsl{\acronym{DR1}}}
% \newcommand{\DRtwo}{\textsl{\acronym{DR2}}}
% \newcommand{\TGAS}{\textsl{\acronym{TGAS}}}
% \newcommand{\DPAC}{{\acronym{DPAC}}}
% \newcommand{\documentname}{\textsl{Note}}
% \newcommand{\equationname}{equation}

% \newcommand{\AU}{\mathrm{A.U.}}
% \newcommand{\dd}{\mathrm{d}}
% \newcommand{\given}{\,|\,}
% \newcommand{\T}{^{\mathsf{T}}}
% \newcommand{\inv}{^{-1}}

\shorttitle{dynamical evidence of a dark-matter substructure}
\shortauthors{bonaca \& hogg}

\begin{document}\sloppy\sloppypar\raggedbottom\frenchspacing

\title{\textbf{%
Encounter of the GD-1 stellar stream with a massive perturber:\\
Direct dynamical evidence of a dark-matter substructure
}}

\correspondingauthor{Ana Bonaca}
\email{ana.bonaca@cfa.harvard.edu}

\author[0000-0002-7846-9787]{Ana Bonaca}
\affil{Harvard--Smithsonian Center for Astrophysics, 60 Garden St, Cambridge, MA 02138, USA}

\author[0000-0003-2866-9403]{David W. Hogg}
\affil{Center for Cosmology and Particle Physics, Department of Physics, New York University, 726~Broadway, New York, NY 10003, USA}
\affil{Center for Data Science, New York University, 60 Fifth Ave, New York, NY 10011, USA}
\affil{Max-Planck-Institut f\"ur Astronomie, K\"onigstuhl 17, D-69117 Heidelberg}
\affil{Flatiron Institute, 162 Fifth Ave, New York, NY 10010, USA}

\begin{abstract}\noindent
The GD-1 stellar stream is a long, thin, cold stream of stars in the Milky Way halo.
It is sensitive to details of halo dynamics, and it has been shown to have structure that is suggestive of non-trivial gravitational interactions in its past.
Here we present a conceptual model for the interaction of GD-1 with a massive perturber that---without fine-tuning---explains many of the features of one of the stream structures, including a gap and an off-stream spur of stars.
The model involves an impulse by a fast encounter, after which the stream grows a loop of stars that is both off the stream track, and on a set of orbits with different frequencies.
The configuration-space observations are most sensitive to a combination of mass, velocity, age, and impact parameter of the encounter, and future velocity-space observations will be sensitive to a different combination of these.
Given sensible assumptions about age and velocity, the perturber must have had a mass in the range $10^6\,\rm M_\odot$ to $10^8\,\rm M_\odot$.
Orbit integrations back in time show that the stream encounter could not have been caused by any known globular cluster or dwarf galaxy, and mass and impact-parameter arguments show that it could not have been caused by a molecular cloud in the Milky Way disk.
The most plausible explanation for the gap-and-spur structure is an encounter with a dark-matter substructure, like those which are naturally predicted in this mass range in this part of the Milky Way halo in \lcdm\ cosmology.
This observation opens up the possibility that detailed observations of streams could measure the mass spectrum of dark-matter substructures and even identify individual substructures and their orbits in the halo.

% We present a conceptual model for the interaction of GD-1, a thin Milky Way stellar stream, with a massive perturber.
% Following the encounter, the stream develops a gap at the location of the closest approach, from which emanates a loop of stream stars.
% Projected on the sky, this loop appears to extend from the stream as a narrow spur, thus reproducing the observed GD-1 morphology.
% Under the impulse approximation, we infer the length scale of impact (?) $GMT/BV = x$ from relative sizes of the gap and spur in GD-1(?).
% Encounters of GD-1 with known satellites and the Galactic disk have $GMT/BV \ll x$.
% Given how GD-1 orbits far from the Galaxy (or some better argument here), the most plausible perturber is a dark matter subhalo, which naturally populate galactic halos in the $\Lambda$CDM cosmology.
% Future kinematic maps of the perturbed region of the GD-1 stream will put constraints on the mass spectrum of dark matter subhalos in the Milky Way.

\end{abstract}

\keywords{%
cosmology:~observations
  ---
dark~matter
  ---
gravitation
  ---
stars:~kinematics~and~dynamics
  ---
Galaxy:~halo
  ---
Galaxy:~kinematics~and~dynamics
}

\section{Introduction}
\label{sec:intro}
- keep short, everything on gd-1 cite
- just mention gap literature, but cite the laundry list of papers

\section{Models of the stream + perturber encounter}
\label{sec:model}
% - literature so far focused on gaps
- to uncover the origin of the gap+spur structure, we analyze a range of idealized encounters between a massive perturber and stars moving on an orbit that fits the GD-1 stream
- following the encounter, stars closest receive an energy kick, and start moving towards the location of the perturber's closest approach
- the single orbit structure of the stream develops a loop after the 
- however, the stars affected by the encounter actually form a loop that extends towards the location of the closest approach (fig)
- connect spur morphology to phases of gap growth from denis
% - sky projections: loop can appear as a spur viewed edge-on

\begin{figure}
\begin{center}
\includegraphics[width=\textwidth]{data_model_comparison.pdf}
\end{center}
\caption{Members of the GD-1 stellar stream, cleanly selected using Gaia proper motions and PanSTARRS photometry, reveal an underdensity in the stream and a spur extending for $\approx10^\circ$ from this gap (top left).
The gap is $\approx8^\circ$ wide and is located at $\phi_1\approx-40^\circ$ (top right).
We present a conceptual model of GD-1 that has been perturbed by a compact, massive object.
The orbital structure of stars closest to the passing perturber is distorted into a loop of stars that after 0.5\,Gyr appears edge-on and matches the extent of the observed spur (bottom left), as well as the location and the width of the gap (bottom right). 
}
\label{fig:model}
\end{figure}

- for some viewing angles, a loop of stream stars formed following an encounter with a massive perturber can appear as a spur and a gap combo
- exact details depend on the following parameters: time of encounter, impact parameter, velocity, mass and size of the perturber
- to find parameters allowed by the data, we would ideally like to forward model, but generative model is hard and we don't have it
- instead, we devised a set of expert criteria that allow us to compare whether a conceptual model represents the observed structure of GD-1
- metrics

\begin{figure}
\begin{center}
\includegraphics[width=\textwidth]{excursions.pdf}
\end{center}
\caption{}
\label{fig:scalings}
\end{figure}

- parameter search
- haven't exhaustively searched the parameter space

\begin{figure}
\begin{center}
\includegraphics[width=0.8\textwidth]{corner.pdf}
\end{center}
\caption{}
\label{fig:corner}
\end{figure}

% - modeled gap, loop looks like a spur
% - scaling rules; show that the configuration-space is most sensitive to $GMT/BV$.
% - show that the velocity-space is most sensitive to $GM/BV$.
% - show that $B$ and angles are next-order effects.
% - are the viewing angles natural for this situation? Might require using a sensible
% orbit for the stream and sensible location for the observer.

\section{Origin of the GD-1 perturber}
\label{sec:origin}
- classes of objects: known satellite, unknown, disk
- integrate backwards to show that it can't be any known object.
- look at molecular cloud properties and show that they are all out.
- give some typical dm substructure numbers for the whole galaxy and for this radius range in the nearby halo.
- also something about the mass-size relationship for these dm structures.
- conclusion: dm substructure is the most plausible of the options.
- interesting point: The predictions about \lcdm\ substructure in the inner halo are controversial right now because of simulation differences. If this matters, this point should enter the abstract and modify the naturalness of the conclusion.

\begin{figure}
\begin{center}
\includegraphics[width=0.8\textwidth]{satellite_distances.pdf}
\end{center}
\caption{Data from \citet{simon2018} and \citet{gdr2_satellites}.}
\label{fig:known_encounters}
\end{figure}

\section{Discussion}
- This is huge!
- Optimistic future of a huge network of cold streams---what might be possible?

\acknowledgements
It is a pleasure to thank
  Kathryn Johnston (Columbia),
  Adrian Price-Whelan (Princeton),
  and
  Hans-Walter Rix (\acronym{MPIA})
for valuable discussions and input.
This project was developed in part at the
2018 \acronym{NYC} \Gaia\ \acronym{DR2} Workshop
at the Center for Computational Astrophysics of the Flatiron Institute
in New York City in 2018 April.

This work has made use of data from the European Space Agency (\acronym{ESA}) mission
\Gaia\ (\url{https://www.cosmos.esa.int/gaia}), processed by the \Gaia\ Data
Processing and Analysis Consortium (\acronym{DPAC},
\url{https://www.cosmos.esa.int/web/gaia/dpac/consortium}). Funding for the
\acronym{DPAC}
has been provided by national institutions, in particular the institutions
participating in the \Gaia\ Multilateral Agreement.

\bibliographystyle{aasjournal}
\bibliography{spur}

\end{document}
