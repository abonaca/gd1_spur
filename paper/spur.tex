% This Note is Copyright 2018 the author.

\documentclass[12pt, modern]{aastex62}

\addtolength{\topmargin}{-0.25in}
\addtolength{\textheight}{0.50in}
\setlength{\parindent}{\baselineskip}

\newcommand{\acronym}[1]{{\small{#1}}}
\newcommand{\Gaia}{\textsl{Gaia}}
% \newcommand{\Tycho}{\textsl{Tycho}}
% \newcommand{\DRone}{\textsl{\acronym{DR1}}}
% \newcommand{\DRtwo}{\textsl{\acronym{DR2}}}
% \newcommand{\TGAS}{\textsl{\acronym{TGAS}}}
% \newcommand{\DPAC}{{\acronym{DPAC}}}
% \newcommand{\documentname}{\textsl{Note}}
% \newcommand{\equationname}{equation}

% \newcommand{\AU}{\mathrm{A.U.}}
% \newcommand{\dd}{\mathrm{d}}
% \newcommand{\given}{\,|\,}
% \newcommand{\T}{^{\mathsf{T}}}
% \newcommand{\inv}{^{-1}}

\shorttitle{direct dynamical evidence for dark-matter substructure}
\shortauthors{bonaca \& hogg}

\begin{document}\sloppy\sloppypar\raggedbottom\frenchspacing

\title{\textbf{%
Encounter of the GD-1 stellar stream with a massive perturber:\\
Direct dynamical evidence of a dark-matter substructure
}}

\correspondingauthor{Ana Bonaca}
\email{ana.bonaca@cfa.harvard.edu}

\author[0000-0002-7846-9787]{Ana Bonaca}
\affil{Harvard--Smithsonian Center for Astrophysics, 60 Garden St, Cambridge, MA 02138, USA}

\author[0000-0003-2866-9403]{David W. Hogg}
\affil{Center for Cosmology and Particle Physics, Department of Physics, New York University, 726~Broadway, New York, NY 10003, USA}
\affil{Center for Data Science, New York University, 60 Fifth Ave, New York, NY 10011, USA}
\affil{Max-Planck-Institut f\"ur Astronomie, K\"onigstuhl 17, D-69117 Heidelberg}
\affil{Flatiron Institute, 162 Fifth Ave, New York, NY 10010, USA}

\begin{abstract}\noindent
We present a conceptual model for the interaction of GD-1, a thin Milky Way stellar stream, with a massive perturber.
Following the encounter, the stream develops a gap at the location of the closest approach, from which emanates a loop of stream stars.
Projected on the sky, this loop appears to extend from the stream as a narrow spur, % on the one side, 
%and along the same line of sight as the unperturbed part of the stream on the other side, 
thus reproducing the observed GD-1 morphology.
Under the impulse approximation, we infer the length scale of impact (?) $GMT/BV = x$ from relative sizes of the gap and spur in GD-1(?).
Encounters of GD-1 with known satellites and the Galactic disk have $GMT/BV \ll x$.
Given how GD-1 orbits far from the Galaxy (or some better argument here), the most plausible perturber is a dark matter subhalo, which naturally populate galactic halos in the $\Lambda$CDM cosmology.
Future kinematic maps of the perturbed region of the GD-1 stream will put constraints on the mass spectrum of dark matter subhalos in the Milky Way.
% - infer age of that part of the stream from distance to the plausible progenitor locations -- thus cap the gap age too

% - tracers of acceleration; cold -> sensitive to perturbations
% - gd-1 extremely non-trivial morphology
% - any of the morphology can be qualitatively and quantitatively explained by an encounter of the stream and grav perturbers?
% - impulse approximation in extremely toy model -- conceptual only (both stream, galaxy, perturber)
% - natural kinds of interactions viewed at natural angle: get gap & spur similar to observed in gd-1
% - to leading order, the morphology seen in gd-1 constrains some of mass, velocity, impact parameter, age
% - however, given the kinematic properties of the halo suggests perturber must be of x msun
% - strong prediction on how velocity should vary along & around the perturbation, plausibly observable ~km/s
% - this mass scale at this gal rad puts constraints on dark matter models
% 
% - we have done the math -- unlikely to have been a gmc or globular cluster
% -- check orbits of clusters
% -- prediction for gmc geometry
\end{abstract}

\keywords{%
cosmology: dark matter --- }

\section{Introduction}
\label{sec:intro}
- keep short, everything on gd-1 cite
- just mention gap literature, but cite the laundry list of papers

\section{Stream + perturber encounter}
\label{sec:model}
- modeled gap, loop looks like a spur
- scaling

\section{Origin of the GD-1 perturber}
\label{sec:origin}
- classes of objects: known satellite, unknown, disk
- conclusion: dm substructure / unknown perturber, more plausible than any of those

\begin{figure}
\begin{center}
\includegraphics[width=\textwidth]{satellite_distances.pdf}
\end{center}
\caption{Data from \citet{simon2018} and \citet{gdr2_satellites}.}
\label{fig:known_encounters}
\end{figure}

\section{Discussion}

\acknowledgements
It is a pleasure to thank
  Kathryn Johnston (Columbia),
  Adrian Price-Whelan (Princeton),
  and
  Hans-Walter Rix (MPIA)
for valuable discussions and input.
This project was developed in part at the
2018 NYC Gaia DR2 Workshop at the Center for Computational Astrophysics of
the Flatiron Institute in New York City in 2018 April.

This work has made use of data from the European Space Agency (ESA) mission
\Gaia\ (\url{https://www.cosmos.esa.int/gaia}), processed by the \Gaia\ Data
Processing and Analysis Consortium (\acronym{DPAC},
\url{https://www.cosmos.esa.int/web/gaia/dpac/consortium}). Funding for the
\acronym{DPAC}
has been provided by national institutions, in particular the institutions
participating in the \Gaia\ Multilateral Agreement.

\bibliographystyle{aasjournal}
\bibliography{spur}

\end{document}
